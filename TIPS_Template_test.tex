\documentclass[]{article}
\usepackage{lmodern}
\usepackage{amssymb,amsmath}
\usepackage{ifxetex,ifluatex}
\usepackage{fixltx2e} % provides \textsubscript
\ifnum 0\ifxetex 1\fi\ifluatex 1\fi=0 % if pdftex
  \usepackage[T1]{fontenc}
  \usepackage[utf8]{inputenc}
\else % if luatex or xelatex
  \ifxetex
    \usepackage{mathspec}
  \else
    \usepackage{fontspec}
  \fi
  \defaultfontfeatures{Ligatures=TeX,Scale=MatchLowercase}
\fi
% use upquote if available, for straight quotes in verbatim environments
\IfFileExists{upquote.sty}{\usepackage{upquote}}{}
% use microtype if available
\IfFileExists{microtype.sty}{%
\usepackage{microtype}
\UseMicrotypeSet[protrusion]{basicmath} % disable protrusion for tt fonts
}{}
\usepackage[margin=1in]{geometry}
\usepackage{hyperref}
\hypersetup{unicode=true,
            pdfborder={0 0 0},
            breaklinks=true}
\urlstyle{same}  % don't use monospace font for urls
\usepackage{graphicx,grffile}
\makeatletter
\def\maxwidth{\ifdim\Gin@nat@width>\linewidth\linewidth\else\Gin@nat@width\fi}
\def\maxheight{\ifdim\Gin@nat@height>\textheight\textheight\else\Gin@nat@height\fi}
\makeatother
% Scale images if necessary, so that they will not overflow the page
% margins by default, and it is still possible to overwrite the defaults
% using explicit options in \includegraphics[width, height, ...]{}
\setkeys{Gin}{width=\maxwidth,height=\maxheight,keepaspectratio}
\IfFileExists{parskip.sty}{%
\usepackage{parskip}
}{% else
\setlength{\parindent}{0pt}
\setlength{\parskip}{6pt plus 2pt minus 1pt}
}
\setlength{\emergencystretch}{3em}  % prevent overfull lines
\providecommand{\tightlist}{%
  \setlength{\itemsep}{0pt}\setlength{\parskip}{0pt}}
\setcounter{secnumdepth}{0}
% Redefines (sub)paragraphs to behave more like sections
\ifx\paragraph\undefined\else
\let\oldparagraph\paragraph
\renewcommand{\paragraph}[1]{\oldparagraph{#1}\mbox{}}
\fi
\ifx\subparagraph\undefined\else
\let\oldsubparagraph\subparagraph
\renewcommand{\subparagraph}[1]{\oldsubparagraph{#1}\mbox{}}
\fi

%%% Use protect on footnotes to avoid problems with footnotes in titles
\let\rmarkdownfootnote\footnote%
\def\footnote{\protect\rmarkdownfootnote}

%%% Change title format to be more compact
\usepackage{titling}

% Create subtitle command for use in maketitle
\newcommand{\subtitle}[1]{
  \posttitle{
    \begin{center}\large#1\end{center}
    }
}

\setlength{\droptitle}{-2em}
  \title{}
  \pretitle{\vspace{\droptitle}}
  \posttitle{}
  \author{}
  \preauthor{}\postauthor{}
  \date{}
  \predate{}\postdate{}

\usepackage{tabu}
\usepackage{graphicx}
\usepackage{multirow}
\usepackage{array}
\renewcommand{\familydefault}{\sfdefault}
\graphicspath{ {S:/Projects/DLM Secure/Psychometrician Asst Projects/Jennifer Projects/} }

\begin{document}

\begin{center}
\textit{\textbf{This is a secure testing document. Do not reproduce or redistribute. Shred after use.}}
\medskip

\begin{tabu} to 1.0\textwidth {  X[l]  X[l]  } 
 \multirow{2}{*}{\includegraphics[height=3cm]{dlm_logo}} & {\textbf{\large{}SP ELA L.3.5.a T R-5347}} \\ [3mm]
 \hfill & {\textbf{Testlet Information Page: ELA5347}} \\ [3mm]
\end{tabu}
\end{center}

\medskip

\hrulefill
\bigskip

\textbf{Testlet Type:} Computer-delivered \medskip

\textbf{Number of Items:} 4 \medskip

\textbf{Materials Needed:} None \medskip

\textbf{Materials Use:} None \medskip

\textbf{Suggested Substitute Materials:} None \medskip

\renewcommand{\arraystretch}{1.5}

\begin{center}
\begin{tabu} to 1.0\textwidth { | X[l]  X[l] | } 
 \hline
 \hfill & \hfill \\
 \textbf{DLM Text Title:} Whales Move & \\ 
 \textbf{Type of Text:} Reading Informational Text  & \textbf{Familiar?} Unfamiliar \\ 
 \textbf{DLM Source Book:} My Father's Dragon  &  \\ 
 \hfill & \hfill \\
 \hline
\end{tabu}
\end{center}

\bigskip

\textbf{Accessibility supports NOT allowed:}

Do not translate words for the student.

Do not define words for the student.

\bigskip

\textbf{Other Comments:} None


\end{document}
